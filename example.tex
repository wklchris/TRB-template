\documentclass[12pt]{trbart}

\begin{document}

\linenumbers%
\maketitle

\section{Abstract}

This section is an abstract. No one would like to write an abstract for a \LaTeX\ template, but you have to write something if you are the author. To add an abstract, use normal \verb+\section+ command is fine. At the end of your abstract, please add a newparagraph with \verb+\trbkeywords{...}+ to show keywords.

\trbkeywords{TRB, template, LaTeX, BibLaTeX}


\section{Introduction}
This documents is an example of the \texttt{trbart} class \LaTeX\ file maintained in \url{https://github.com/wklchris/TRB-template}. I would introduce main settings of the template through this document.

This document is for developers (users do \textbf{NOT} need to read this document thoroughly, but you might find something valuable if you flip through it). If you are a user who doesn't care details at all, please refer to the ReadMe file as a quick guide.


\section{Outline Formats}
This template is based on the standard \texttt{article} documnetclass, and there are 3 outline levels in total:
\begin{itemize}
    \item Section: Each letter is captalized. Bold series.
    \item Subsection: Initial letter of each word is captalized. Bold series.
    \item Subsubsection: Only first letter of the whole title is captalized. Italic shape.
\end{itemize}

Here are examples to show the format of different outline levels.

\subsection{A subsection example}
This is a subsection example.

\subsubsection{A subsubsection example}
This is a subsubsection example.


\section{Titlepage}
This template does not use the original command for titlepage (like \verb+\title+). Instead, it uses \verb+\trbtitle+. \textbf{You should capitalize initial letter of each word in the title manually} (and pay attention to `a', `an', `the', etc.), since this template doesn't automatically do the capitalization for you.

\subsection{Authors}
The titlepage is inserted by the command \verb+\maketitle+ (which has been redefined). Author names are in bold font (\verb+\authors+), followed by the affiliation of the author (\verb+\affils+), the city \& state with zipcode (\verb+\addresses+), and his/her email (\verb+\emails+). 

To modify aforementioned informations in the titlepage, I suggest you to open \texttt{trbart.cls} file and search for ``trbtitle''. You will see all of their definitions there and then you can edit them accordingly. 

\subsection{Word Count}
This template cannot give any suggestion on counting word automatically while compiling. Users are expected to use external tools to count words in their project. 

For offline LaTeX users, a simple way is to use the \texttt{texcount} CMD/Terminal command, which is already provided by the TeX Live distribution:
\begin{verbatim}
texcount example.tex -utf8 -inc -incbib -sum -1 
\end{verbatim}
This command will return the total number of words in the \texttt{example.tex} file, including words in all subfiles and the bibliography section.

For online users, Overleaf also provides such feature. Users can open your project, click on the ``Menu'' and find a ``Word Count'' button there. As my personal experience, word count from Overleaf seems to be slightly lower than the number given by the \texttt{texcount} command.

When complete word count, you can open the \texttt{trbart.cls} file to edit the number of words. And do not forget to count the number of your tables!


\section{Equations}
Equations are left-aligned and counted towards line numbers. This style is achieved by \texttt{fleqn} option of the \texttt{amsmath} package with \verb+\mathindent+ set to zero. Please use \texttt{equation} environment to write equations so that they can be correctly linenumbered --- I only configured the compatiblity between this environment and the \texttt{fleqn} option. Let's try a well-known theorem:
\begin{equation}\label{eq:triangle}
    a^2 + b^2 = c^2
\end{equation}
where \(c\) is the hypotenuse of a right triangle, \(a\) and \(b\) are the two legs.

Here is a more complicated example. Though there are two lines, but the this equation will be counted as whole in the line number sequence.
\begin{equation}\label{eq:exp}
\begin{dcases}
    \frac{\ud \boldsymbol{p}}{\ud t} &= -\pfrac{H}{\boldsymbol{q}} \\
    \frac{\ud \boldsymbol{q}}{\ud t} &= \pfrac{H}{\boldsymbol{p}}
\end{dcases} 
\end{equation}

The citation style of equations (\autoref{eq:triangle} and \autoref{eq:exp}) are bold with the default color.

\section{Floatings: Figures and Tables}
There is nothing special for figures. \autoref{fig:longtitle} is an example of a figure with a long title.
\begin{figure}[!hbt]
    \centering
    \includegraphics[width=0.7\textwidth]{example-image-a}
    \caption{This is a figure with a very long caption because I want to show you how the indentation works when there is a line break in a figure's title.\\ I'm not sure about the indent for titles with more than one paragraph.}\label{fig:longtitle}
\end{figure}

You may include table with \texttt{tabularx} package for complex tables. The \LaTeX\ code below will create an output of \autoref{tab:shorttitle}. Table typesetting varies from person to person, so you don't have to follow my settings.

The \LaTeX\ code of \autoref{tab:shorttitle} is shown below:
\begin{verbatim}
\begin{table}[!hbt]
\centering
\caption{A table with bold and center-aligned headers}
\label{tab:shorttitle}
%\newcolumntype{E}{>{\raggedleft\arraybackslash}X}
\begin{tabularx}{0.5\textwidth}{crEE}
    \toprule
    \multirowthead{2}{Obs} & \multirowthead{2}{Attr 1} &
    \multicolumn{2}{c}{\thead{Attr 2}} \\
    \cmidrule(lr){3-4}
    & & \thead{SubAttr 1} & \thead{SubAttr 2} \\
    \midrule
    A & 1.0 & 2.0 & -12.0 \\
    B & -3.0 & -101.4 & 4.7 \\
    \bottomrule
\end{tabularx}
\end{table}   
\end{verbatim}

\begin{table}[!hbt]
    \centering
    \caption{A table with bold and center-aligned headers}\label{tab:shorttitle}
    \begin{tabularx}{0.5\textwidth}{crEE}
        \toprule
        \multirowthead{2}{Obs} & \multirowthead{2}{Attr 1} & \multicolumn{2}{c}{\thead{Attr 2}} \\
        \cmidrule(lr){3-4}
        & & \thead{SubAttr 1} & \thead{SubAttr 2} \\
        \midrule
        A & 1.0 & 2.0 & -12.0 \\
        B & -3.0 & -101.4 & 4.7 \\
        \bottomrule
    \end{tabularx}
\end{table}

We can refer to equations or floatings using \verb+\autoref+ commands, showing as \autoref{eq:exp}, \autoref{fig:longtitle} and \autoref{tab:shorttitle}. The bold format is achieved because the \verb+\autoref+ command has been redefined in the template.

\section{References Section}
Different with normal settings, this template requires \texttt{biblatex} package with Biber backend, while a large amount of users are using BibTeX backend. These two approaches both accept \texttt{.bib} file as the references source, but with some differences in entry (key) names of some references type. For example, the biblatex package uses \texttt{journaltitle} for \texttt{\@article} type, instead of BibTeX's \texttt{journal}. 

Here is a simple table showing the entry types that works with biblatex package (\autoref{tab:shorttitle}). This time, the table is included with the default \texttt{tabular} environment, but not \texttt{tabularx}.
\begin{table}[!hbt]
    \centering
    \caption{Common entry types in biblatex's bib files}\label{tab:bib}
    \begin{tabular}{rlll}
        \toprule
        \thead{Doc types} & \thead{Alias} & \thead{Keys} & \thead{Also often used} \\
        \midrule
        article & & author, title, journaltitle, year/date & \\
        book & & author, title, year/date & publisher \\
        online & electronic & author/editor, title, year/date, url & urldate \\
        inproceedings & conference & author, title, booktitle, year/date & location \\
        thesis & \({}^\dagger{}\) & author, title, type, institution, year/date & \\
        \bottomrule
        \(\dagger{}\) & \multicolumn{3}{l}{You need to use `type' key to specify it is a master's or phd's thesis;} \\
        & \multicolumn{3}{l}{or use alias masterthesis/phdthesis without `type' key.}
    \end{tabular}
\end{table}

The references section starts from a new page. TRB allows numeric citation or Harvard style citation. This article goes with a numeric citation style with round brackets. For example, here I cite an article~\autocite{egarticle}, a book~\autocite{egbook}, and a conference article~\autocite{egconference}. Sometimes we also cite online files~\autocite{egonline}. 

In this template, formats of bibliography items are similar to:
\begin{description}
    \item[article] Last, First, Last, First, and Last, First. Title. \textit{Journaltitle}, year. Volume: doi or pagerange.
    \item[book] Authors. \textit{Title}. Publisher, year.
    \item[online] Authors. \textit{Title}. \texttt{URL}. Accessed July 1, 1900.
    \item[inproceedings] Authors. Title. \textit{Booktitle}. Location, year.
    \item[thesis] Authors. \textit{Title}. Institution, year. 
\end{description} 

You can see the references list on the next page. The list is given by a stated command named \verb+\printtrbrefs+.

% Reference
\printtrbrefs%

\end{document}